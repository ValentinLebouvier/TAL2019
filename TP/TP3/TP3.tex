% Created 2019-06-11 mar. 13:40
\documentclass{article}
\usepackage[utf8]{inputenc}
\usepackage[T1]{fontenc}
\usepackage{fixltx2e}
\usepackage{graphicx}
\usepackage{longtable}
\usepackage{float}
\usepackage{wrapfig}
\usepackage{rotating}
\usepackage[normalem]{ulem}
\usepackage{amsmath}
\usepackage{textcomp}
\usepackage{marvosym}
\usepackage{wasysym}
\usepackage{amssymb}
\usepackage{hyperref}
\tolerance=1000
\usepackage[frenchb]{babel}
\author{Valentin LEBOUVIER}
\date{\today}
\title{TP3 : MVC}
\hypersetup{
  pdfkeywords={},
  pdfsubject={},
  pdfcreator={Emacs 25.2.2 (Org mode 8.2.10)}}
\begin{document}

\maketitle

\section{MVC}
\label{sec-1}
(Explication du MVC)

\section{Un PacMan revisité}
\label{sec-2}
Lors de ce TP vous aurez à réaliser une vue associée à cet ensemble modèle/controlleur:

(diagramme de classe modèle/controlleur)
% Emacs 25.2.2 (Org mode 8.2.10)
\end{document}