% Created 2019-06-11 mar. 10:31
\documentclass{article}
\usepackage[utf8]{inputenc}
\usepackage[T1]{fontenc}
\usepackage{fixltx2e}
\usepackage{graphicx}
\usepackage{longtable}
\usepackage{float}
\usepackage{wrapfig}
\usepackage{rotating}
\usepackage[normalem]{ulem}
\usepackage{amsmath}
\usepackage{textcomp}
\usepackage{marvosym}
\usepackage{wasysym}
\usepackage{amssymb}
\usepackage{hyperref}
\tolerance=1000
\author{Valentin LEBOUVIER}
\date{\today}
\title{TP1 : Graphes et Behaviour Trees}
\hypersetup{
  pdfkeywords={},
  pdfsubject={},
  pdfcreator={Emacs 25.2.2 (Org mode 8.2.10)}}
\begin{document}

\maketitle
\tableofcontents


\section{Rappels sur les graphes:}
\label{sec-1}
Rappels sur les graphes: (non-)orientés, (a)cycliques

Parcours largeur, hauteur \ldots{}


\section{Graphes simples:}
\label{sec-2}
Noeuds (init(name), addChild, addChildren, parcoursLargeur, parcourProfondeur)

Avec affichage du nom du noeud

\section{Behaviour trees:}
\label{sec-3}
Présentation BT orienté IA/ jeux vidéo
Présentation des différents types de noeuds \& fonctionnement global

Construction example simple avec py$_{\text{trees}}$
% Emacs 25.2.2 (Org mode 8.2.10)
\end{document}