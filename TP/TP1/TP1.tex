% Created 2019-06-18 mar. 11:47
\documentclass{article}
\usepackage[utf8]{inputenc}
\usepackage[T1]{fontenc}
\usepackage{fixltx2e}
\usepackage{graphicx}
\usepackage{longtable}
\usepackage{float}
\usepackage{wrapfig}
\usepackage{rotating}
\usepackage[normalem]{ulem}
\usepackage{amsmath}
\usepackage{textcomp}
\usepackage{marvosym}
\usepackage{wasysym}
\usepackage{amssymb}
\usepackage{hyperref}
\tolerance=1000
\usepackage[frenchb]{babel}
\date{\today}
\title{TP1 : Graphes et Behaviour Trees}
\hypersetup{
  pdfkeywords={},
  pdfsubject={},
  pdfcreator={Emacs 25.2.2 (Org mode 8.2.10)}}
\begin{document}

\maketitle

\section{Objectif des TPs}
\label{sec-1}
Ce TP est le premier d'une série de trois TPs dont le but final est de créer une interface pour un jeu de PacMan.
Dans ce TP, vous verrez un outils qui s'apelle les Behaviour Trees.
C'est un outil qui sert dans les domaines du jeu-vidéo et de la robotique pour contrôler les agents autonomes (i.e. NPC, robot,\ldots{}).
Dans le second TP, vous apprendrez à créer une interface tkinter.
Et dans le dernier TP, vous mettrez en place une interface pour un PacMan.
Et si vous êtes rapides, vous pourrez même créer des Behaviour trees pour contrôler le PacMan comme vous le voulez.

\section{Rappels sur les graphes:}
\label{sec-2}
Les graphes sont des structures de données

Rappels sur les graphes: (non-)orientés, (a)cycliques

Parcours largeur, hauteur \ldots{}


\section{Graphes simples:}
\label{sec-3}
Noeuds (init(name), addChild, addChildren,  parcourProfondeur)

Avec affichage du nom du noeud lors de la visite en profondeur

\section{Behaviour trees:}
\label{sec-4}
Présentation BT orienté IA/ jeux vidéo

Présentation des différents types de noeuds \& fonctionnement global

Construction exemple simple avec py-trees
% Emacs 25.2.2 (Org mode 8.2.10)
\end{document}