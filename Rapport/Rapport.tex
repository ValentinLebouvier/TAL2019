% Created 2019-05-23 jeu. 15:23
\documentclass[11pt]{article}
\usepackage[utf8]{inputenc}
\usepackage[T1]{fontenc}
\usepackage{fixltx2e}
\usepackage{graphicx}
\usepackage{longtable}
\usepackage{float}
\usepackage{wrapfig}
\usepackage{rotating}
\usepackage[normalem]{ulem}
\usepackage{amsmath}
\usepackage{textcomp}
\usepackage{marvosym}
\usepackage{wasysym}
\usepackage{amssymb}
\usepackage{hyperref}
\tolerance=1000
\author{ValentinLebouvier}
\date{\today}
\title{Behaviour Trees}
\hypersetup{
  pdfkeywords={},
  pdfsubject={},
  pdfcreator={Emacs 25.2.2 (Org mode 8.2.10)}}
\begin{document}

\maketitle
\tableofcontents


\section{Définition}
\label{sec-1}
\subsection{Éthymologie}
\label{sec-1-1}
Un Behaviour Tree se traduirait en francais par ``Arbre de comportement´´
\subsection{Utilité}
\label{sec-1-2}
Un behaviour tree est un outil permettant de modéliser des scénarios de choix.
% Emacs 25.2.2 (Org mode 8.2.10)
\end{document}